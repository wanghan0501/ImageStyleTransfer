%!TEXTS-program = xelatex
%!TEX encoding = UTF-8 Unicode
\documentclass[a4paper, 11pt]{article}

%%%%%% 导入包 %%%%%%
\usepackage{xeCJK}
\usepackage{graphicx}
\usepackage[unicode]{hyperref}
\usepackage{xcolor}
\usepackage{cite}
\usepackage{indentfirst}
\usepackage{amsmath}
\usepackage{amssymb}
\usepackage{float}

%%%%%% 设置字号 %%%%%%
\newcommand{\chuhao}{\fontsize{42pt}{\baselineskip}\selectfont}
\newcommand{\xiaochuhao}{\fontsize{36pt}{\baselineskip}\selectfont}
\newcommand{\yihao}{\fontsize{28pt}{\baselineskip}\selectfont}
\newcommand{\erhao}{\fontsize{21pt}{\baselineskip}\selectfont}
\newcommand{\xiaoerhao}{\fontsize{18pt}{\baselineskip}\selectfont}
\newcommand{\sanhao}{\fontsize{15.75pt}{\baselineskip}\selectfont}
\newcommand{\sihao}{\fontsize{14pt}{\baselineskip}\selectfont}
\newcommand{\xiaosihao}{\fontsize{12pt}{\baselineskip}\selectfont}
\newcommand{\wuhao}{\fontsize{10.5pt}{\baselineskip}\selectfont}
\newcommand{\xiaowuhao}{\fontsize{9pt}{\baselineskip}\selectfont}
\newcommand{\liuhao}{\fontsize{7.875pt}{\baselineskip}\selectfont}
\newcommand{\qihao}{\fontsize{5.25pt}{\baselineskip}\selectfont}

%%% 英文相关字体属性 %%%%
% \usepackage{fontspec,xltxtra,xunicode}
% \defaultfontfeatures{Mapping=tex-text}
% \setromanfont[Mapping=tex-text]{Hoefler Text}
% \setsansfont[Scale=MatchLowercase,Mapping=tex-text]{Gill Sans}
% \setmonofont[Scale=MatchLowercase]{Andale Mono}

%%%% 设置 font 属性 %%%%
\setCJKmainfont[BoldFont ={STXihei},ItalicFont ={STKaiti}]{STFangsong}%{STSong}  %设置中文正体字体,BoldFont设置粗体和斜体样式对应的字体
\setCJKsansfont{STXihei}%设置无衬线样式对应字体
\setCJKmonofont{STFangsong} %设置有衬线样式对应字体
\punctstyle{hangmobanjiao} %行末半角式:所有标点占一个汉字宽度,行首行末对齐

%%%% 设置 section 属性 %%%%
\makeatletter
\renewcommand\section{\@startsection{section}{1}{\z@}%
{-1.5ex \@plus -.5ex \@minus -.2ex}%
{.5ex \@plus .1ex}%
{\normalfont\sihao\CJKfamily{hei}}}
\makeatother

%%%% 设置 subsection 属性 %%%%
\makeatletter
\renewcommand\subsection{\@startsection{subsection}{1}{\z@}%
{-1.25ex \@plus -.5ex \@minus -.2ex}%
{.4ex \@plus .1ex}%
{\normalfont\xiaosihao\CJKfamily{hei}}}
\makeatother

%%%% 设置 subsubsection 属性 %%%%
\makeatletter
\renewcommand\subsubsection{\@startsection{subsubsection}{1}{\z@}%
{-1ex \@plus -.5ex \@minus -.2ex}%
{.3ex \@plus .1ex}%
{\normalfont\xiaosihao\CJKfamily{hei}}}
\makeatother

%%%% 段落首行缩进两个字 %%%%
\makeatletter
\let\@afterindentfalse\@afterindenttrue
\@afterindenttrue
\makeatother
\setlength{\parindent}{2em}  %中文缩进两个汉字位


%%%% 下面的命令重定义页面边距,使其符合中文刊物习惯 %%%%
\addtolength{\topmargin}{-54pt}
\setlength{\oddsidemargin}{0.63cm}  % 3.17cm - 1 inch
\setlength{\evensidemargin}{\oddsidemargin}
\setlength{\textwidth}{14.66cm}
\setlength{\textheight}{24.00cm}    % 24.62

%%%% 下面的命令设置行间距与段落间距 %%%%
\linespread{1.4}
% \setlength{\parskip}{1ex}
\setlength{\parskip}{0.5\baselineskip}

%%%% 正文开始 %%%%
\begin{document}

%%%% 定理类环境的定义 %%%%
\newtheorem{example}{例}             % 整体编号
\newtheorem{algorithm}{算法}
\newtheorem{theorem}{定理}[section]  % 按 section 编号
\newtheorem{definition}{定义}
\newtheorem{axiom}{公理}
\newtheorem{property}{性质}
\newtheorem{proposition}{命题}
\newtheorem{lemma}{引理}
\newtheorem{corollary}{推论}
\newtheorem{remark}{注解}
\newtheorem{condition}{条件}
\newtheorem{conclusion}{结论}
\newtheorem{assumption}{假设}

%%%% 重定义 %%%%
\renewcommand{\contentsname}{目录}  % 将Contents改为目录
\renewcommand{\abstractname}{摘要}  % 将Abstract改为摘要
\renewcommand{\refname}{参考文献}   % 将References改为参考文献
\renewcommand{\indexname}{索引}
\renewcommand{\figurename}{图}
\renewcommand{\tablename}{表}
\renewcommand{\appendixname}{附录}
\renewcommand{\algorithm}{算法}


%%%% 定义标题格式,包括title,author,affiliation,email等 %%%%
\title{图像处理报告}
\author{王晗\footnote{电子邮件: hanwang.0501@gmail.com,学号: 2014141463191}\\[2ex]
\xiaosihao 四川大学吴玉章学院\\[2ex]
}
\date{2017年11月}


%%%% 以下部分是正文 %%%%
\maketitle

\tableofcontents
\newpage
\section{传统方法}

\section{神经网络方法}
\subsection{风格迁移}
原始的风格迁移\footnote{论文名:A neural algorithm of artistic style,论文地址:https://arxiv.org/pdf/1508.06576v2.pdf}的速度是非常慢的。在GPU上,生成一张图片都需要10分钟左右,而如果只使用CPU而不使用GPU运行程序,甚至需要几个小时。这个时间还会随着图片尺寸的增大而迅速增大。这其中的原因在于,在原始的风格迁移过程中,把生成图片的过程当做一个“训练”的过程。每生成一张图片,都相当于要训练一次模型,这中间可能会迭代几百几千次,从头训练一个模型要比执行一个已经训练好的模型要费时太多。而这也正是原始的风格迁移速度缓慢的原因。
\subsection{快速风格迁移}
快速风格转移很好的解决了原始风格迁移速度缓慢的问题,它不把生成图片当做一个“训练”的过程,而当成一个“执行”的过程。

\end{document}
